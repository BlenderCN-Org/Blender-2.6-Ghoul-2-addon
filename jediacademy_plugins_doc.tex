\documentclass[a4paper,10pt]{article}

%opening
\title{Blender Jedi Academy Plugin Suite Manual}
\author{Willi ``mrwonko'' Schinmeyer}

\begin{document}
 
 \maketitle
 
 \newpage
 
 \tableofcontents
 
 \newpage
 
 \section{Purpose of this document}
 
 This manual is supposed to explain the various Blender plugins for Jedi Academy file formats I've worked on.
 It is not supposed to explain how to use Blender, there's plenty of documentation on that already.
 
 \section{Changelog}
 
 \subsection*{0.1 (\date{2012-10-28})}
 
 Initial Release for Blender 2.64. Includes:
 \begin{itemize}
  \item ASE Exporter (static geometry)
  \item ASE Importer (static geometry)
  \item GLA Exporter (animations)
  \item GLA Importer (animations)
  \item GLM Exporter (animated (player)models)
  \item GLM Importer (animated (player)models)
  \item MD3 Exporter (static/per-vertex animated geometry)
  \item ROFF Exporter (rigid body animation)
  \item ROFF Importer (rigid body animation)
 \end{itemize}
 
 \subsection*{0.2 (\date{2013-03-13})}
 Mostly a bugfix release:
 \begin{itemize}
  \item Fixed an implicit int-to-bool conversion in the GLM importer that made Blender 2.65+ cringe
  (i.e. fixed GLM import in 2.65+)
  \item Possibly fixed a bug that might've lead to incorrectly imported tags on GLM files
  \item Fixed a bug where new skeletons/animations (not based on existing ones) would not export correctly
  due to an invalid inverted base matrix.
  \item Fixed a bug where ROFF import did not work due to looking for files with an .ase extension
  \item Fixed a bug where ROFF export would append .ase instead of .rof
  \item Added UV support to ASE import (possibly broke import for some faces in very rare cases?
  Report if there's a missing face on import.)
  \item ROFF Export scale defaults to 100\% now
  \item Wrote this manual (yay!)
 \end{itemize}
 
 \section{Installation}
 
 Open the User Preferences (File/User Preferences or Ctrl+Alt+U), go to the Addons tab and press there
 ``Install from File'' button. Select the zip archive containing the addon. If you wish to manually install the
 addon, refer to the Add-Ons section of the Blender Manual for the correct paths.
 
 Once installed, don't forget to enable the addon. It's located in the Import-Export category and called
 ``Import-Export: Jedi Academy Import/Export Tools''. Then click Save User Settings so it'll still be activated
 the next time you start Blender.
 
 \section{ASE}
 
 The ASE file format (Ascii Scene Export) is a plaintext model format that is widely used in Jedi Academy due
 to its simplicity and 3ds Max's ability to export it out of the box.
 
 The Plugin supports only a subset of the features the ASE format has, but so does Q3Map2. Everything important
 is there, unless normals are important to you.
 
 \subsection{ASE Exporter}
 
 Located in ``File/Export/JA ASE (.ase)'' or invoked by entering ``Export JA ASE (.ase)'' in the Search Window
 opened by default with Space.
 
 Things to keep in mind:
 \begin{itemize}
  \item All mesh objects in the current scene will be exported.
  \item The meshes have to be UV mapped.
  \item The meshes should be triangulated -- although no error will be raised if they're not, only the first
  3 vertices of each face will be exported.
  \item The texture used will be the name of the first material on the object. Multi-material objects are not
  supported.
  \item Normals are not exported.
  \item Models do not need to be split at UV seams, Q3Map2 will take care of that.
 \end{itemize}
 
 \subsection{ASE Importer}
 
 Located in ``File/Import/JA ASE (.ase)'' or invoked by entering ``Import JA ASE (.ase)'' in the Search Window
 opened by default with Space.
 
 Not much to say about this one... Imports meshes with uv coordinates from an .ase file, ignoring pretty much
 everything else, as does Q3Map2. Textures get applied as material names.
 
 \section{ROFF}
 
 ROFF is short for Rotation Origin File Format. And it's pretty much just that: It saves the rotation and
 origin/position of an object each frame, relative to the previous position/rotation. Can be used to animate
 rigid objects like brushes (or bmodels, more specifically) via a script (like the boulder in t3\_rift), as
 well as for camera paths.
 
 One thing to note is that the framerate is saved as milliseconds per frame, so rounding may occur. The
 files shipped with Jedi Academy all have 50ms/frame, which is exactly 20 FPS.
 
 \subsection{ROFF Exporter}
 
 Located in ``File/Export/JA ROFF (.rof)'' or invoked by entering ``Export JA ROFF (.rof)'' in the Search Window
 opened by default with Space.
 
 Features:
 \begin{itemize}
  \item Exports location/rotation of the currently selected object over time, so select exactly one object.
  \item Uses Blender's start/end frames to determine the animation's length.
  \item Uses Blender's FPS setting from the Render Settings.
  \item Converts 1:1 between Blender Unit and Jedi Academy's Units, unless you supply a different scale in
  the export settings.
 \end{itemize}

 \subsection{ROFF Importer}
 
 Located in ``File/Import/JA ROFF (.rof)'' or invoked by entering ``Import JA ROFF (.rof)'' in the Search Window
 opened by default with Space.
 
 Features:
 \begin{itemize}
  \item Imports location/rotation for the currently selected object over time, so select exactly one object.
  \item Sets Blender's start/end frames to match the animation's length.
  \item Sets Blender's FPS setting in the Render Settings to match the one in the file.
 \end{itemize}
 
 \section{MD3}
 
 Originally created for Quake 3, the MD3 format is for models, with the possibility of per-vertex-animations
 (which make for huge files). It notably uses fixed-point coordinates for the vertices, resulting in the
 requirement that models need to stay within the [-512; 512) range and only have a precision of 1/64 unit,
 and support so-called Tags to mark locations on the model (e.g. muzzle flash position/rotation on a weapon).
 
 If you want animated map models, this should be the format of your choice.
 
 \subsection{MD3 Exporter}
 
 Located in ``File/Export/JA MD3 (.md3)'' or invoked by entering ``Export MD3'' in the Search Window
 opened by default with Space.
 
 Acknowledgement: I did not originally write this exporter, I improved upon one I found online which was at
 the time maintained by Xembie, although I believe he did not initially start it. His version has since been
 updated and the current version is available on\\
 http://sourceforge.net/projects/md3exporter/
 
 Features:
 \begin{itemize}
  \item Exports \emph{animated} MD3 models. So if you don't want animations, make sure to set the Blender
  End Frame to the same as the Start Frame or your model will turn out needlessly large.
  \item Looks for a custom property called ``md3shader'' on the \emph{objects} to determine the texture to use.
  (Make sure not to put the custom property on the mesh.) 
  \item Exports only selected objects.
  \item Mesh objects get exported as meshes.
  \item Empty objects get exported as tags.
  \item Can be scaled and offset using the corresponding Export settings.
  \item The ``MD3 Name'' Export setting should be set to the path of your model including the extension, e.g.\\
  ``models/map\_objects/imperial/field\_post.md3''.
  \item Make sure your model does not exceed the [-512; 512) range or you'll get a struct.error on export.
 \end{itemize}

 
 \section{Ghoul 2 (GLM/GLA)}
 
 The Ghoul 2 format is Raven Software's skeleton-based model/animation file format used for playermodels in
 Jedi Academy and its other Quake 3 Engine games. GLM files are Ghoul 2 Models, which are saved separately from
 the GLA files, the Ghoul 2 Animations, which can be shared amongst models.
 
 
 \subsection{Features of the File Format}
 \begin{itemize}
  \item Levels of Detail (LOD) -- less expensive versions of the model for use at greater distances and on
  slower machines.
  \item Tags -- placeholders marking locations on the model, e.g. where weapons should be held. Represented by
  single triangle meshes so they can be skinned to the armature.
  \item Skins -- not a Ghoul 2 feature per se, but tightly connected: A file describing what textures to put on
  which surface.
 \end{itemize}
 
 
 \subsection{Limits}
 
 Due to the way data is saved there are some inherent limits for Ghoul 2 files:
 \begin{itemize}
  \item The Mesh must be triangulated.
  \item Every Vertex has to weighted to the skeleton.
  \item A Mesh can reference no more than 32 bones.
  \item Any single vertex can reference no more than 4 out of those 32 bones.
  \item Names must be no longer than 64 characters (bones, objects etc.).
  \item UV coordinates are per-vertex, meaning that any single vertex can only have one UV coordinate. This
  necessitates splitting the mesh at UV seams, which can for example be achieved by also marking seams as
  ``sharp'' and using the ``Edge Split'' modifier to split at those sharp edges.
  \item Bones must not be scaled. The file format only saves location and rotation and applying scale anyway
  while lead to (visually) broken animations.
  \item Bone must not move further than 512 units away from their base pose.
  \item Objects that are to be exported need the Ghoul 2 Custom Properties
  \item LODs are similar: Lower Levels of Detail (with higher indices) can have no additional Meshes, and the
  Meshes that are present will use the same texture as the ones on LOD 0.
 \end{itemize}
 Jedi Academy imposes some additional limits, most notably a maximum of 1000 vertices per object.
 
 \subsection{Scene Setup}
 
 The exporter uses a very specific setup to determine the Ghoul 2 specific settings. The Importer creates
 such a hierarchy, so you can reexport an imported model without any changes.
 
 \subsubsection{Hierarchy}
 
 The root object is called ``scene\_root'' and will likely be an Empty. Its children are the Armature, called
 ``skeleton\_root'', if any, and the different LODs, name ``model\_root\_\emph{[LOD]}'', where [LOD] is the Level
 of Detail starting at 0 for the most detailled one. The actual hierarchy below that only matters for LOD 0,
 because the others mirror it. ``model\_root\_0'' should have a single child, which corresponds to the root
 object in the GLM, and below that the hierarchy is up to you, as far as the exporter is concerned (for proper
 dismemberment in Jedi Academy you'll need to adhere to a certain pattern though, look at existing  models
 for reference).
 
 \subsubsection{Custom Properties}
 
 To be able to match the objects across LODs they need the same name. This is impossible in Blender, object
 names are unique, so the exporter uses a custom property instead. (In theory a prefix would be possible as
 well, this solution is a leftover from the olden days when names were severly limited in length.) For easier
 management of the custom properties the Plugin adds a new Panel to the Object Properties called ``Ghoul 2
 Properties''. For an object without the required Custom Properties it consists of a ``Add G2 properties'' button.
 Press it to add the properties.
 
 Note that for all LODs except the first one (LOD 0), only the name is used since the properties are
 the same across all LODs, which is why the matching of objects is necessary in the first place. This
 especially means that the surfaces will use the same texture!
 
 There are the following properties:
 
 \paragraph*{name}
 The surface's name, for matching across LODs.
 
 \paragraph*{shader}
 The material to apply to this surface. Ignored in favor of skin files in Jedi Academy.
 
 \paragraph*{tag}
 Whether this surface is a Tag, i.e. represents a position, e.g. where a weapon should be held. The most
 important part about a Tag are the first 3 vertices, although there has to be at least one triangle. Vertex 2
 is the origin, Vertex 0 the direction of the X-Axis and Vertex 1 the direction of the Y-Axis, although some
 math is involved to guarantee they're perpendicular. In Raven's models, the short leg is the X Axis, the
 long one the Y Axis. (Unless they screw up, as in R2D2's case.)
 
 \paragraph*{tag}
 Whether this surface should be hidden. Mostly ignored though -- ModView looks for \_off suffixes in the names
 while Jedi Academy consults the skin files.
 
 
 \subsection{GLM Import}
 
 Located in ``File/Import/JA Ghoul 2 model (.glm)'' or invoked by entering ``Import JA Ghoul 2 model (.glm)''
 in the Search Window opened by default with Space.
 
 \subsubsection*{Settings}
 
 \paragraph*{Skin}
 If you'd like to load the textures from a skin file, enter its name here. If the model is called ``model.glm''
 and you'd like to use ``model\_blue.skin'', just enter ``blue''.
 
 \paragraph*{Guess Textures}
 Since Jedi Academy uses skin files, the actual textures specified in the GLM are ignored. It's probably for
 that reason that their first character is replaced by the end-of-string symbol. Guessing the original first
 character is not so hard if the path is ``?odels/players/...'' though, so tick this box to make the importer
 try to figure it out. You're advised to just use the Skin setting instead though.
 
 \paragraph*{Base Path}
 The importer tries to load the textures and apply them to the model, as well as the matching .gla. To be able
 to do that, it needs to know where to look. It will try to examine the path of the opened file, looking for
 ``GameData'' in it. If you're using a non-standard folder layout that may fail, so you can enter a custom path
 to be prepended to textures when searching.
 
 \paragraph*{.gla override}
 GLM files reference the GLA they're for (for Jedi Academy's playermodels that's
 ``models/players/\_humanoid/\_humanoid.gla'') and the importer loads the skeleton as well. Here you can set
 a different GLA to use instead, using Jedi Academy's base-relative notation (e.g.\\
 ``models/players/\_humanoid\_kor1/\_humanoid\_kor1.gla''). This will only work if the number of bones matches.
 
 \paragraph*{Scale}
 A scale to apply, since using a 1:1 conversion from Jedi Academy units (roughly an inch each) to Blender
 (typically a meter) results in huge models. This only rescales the scene\_root, and since export is done
 relative to that this setting will automatically be ignored when exporting.
 
 \paragraph*{Skeleton changes}
 Jedi Academy uses a Skeleton/Armature that's a little strange in places. For example the lower arms are not
 children of the upper arms, making animation quite the pain. The plugin can automatically correct the
 default skeleton in that regard if you select ``Jedi Academy \_humanoid'' here. Animations will still be
 properly applied and exporting works with a changed hierarchy, too -- the plugin will do the required math.
 
 \paragraph*{Animations}
 If you want to import animations, make sure not to set this to ``None''. Importing all frames takes forever
 though, and Blender can't handle it very well either -- that's because in GLA files every frame is a keyframe
 and there's over 20000 of them in Jedi Academy's \_humanoid.gla. For that reason there's an option to import
 only a range.
 
 \paragraph*{Start frame}
 If you choose to only import a range of frames (and make sure to choose that in the Animations setting if
 this is your intention, this setting on its own won't do anything!), this sets at which frame to start, 
 counting from 0.
 
 \paragraph*{Number of frames}
 If you choose to only import a range of frames, this sets how many to import.
 
 
 \subsection{GLA Import}
 
 Located in ``File/Import/JA Ghoul 2 skeleton/animation (.gla)'' or invoked by entering ``Import JA Ghoul 2
 skeleton (.glm)'' in the Search Window opened by default with Space.
 
 \subsubsection*{Settings}
 
 Refer to the GLM Import settings, since these are a subset of them.
 
 
 \subsection{GLM Export}
 
 Located in ``File/Export/JA Ghoul 2 model (.glm)'' or invoked by entering ``Export JA Ghoul 2 model (.glm)''
 in the Search Window opened by default with Space.
 
 \subsubsection*{Settings}
 
 \paragraph*{Base Path}
 See GLM Import Base Path setting.
 
 \paragraph*{.gla name}
 Which skeleton this model uses, since internall no bone names are used, just indices, so the exporter has
 to look them up. Use Jedi Academy's base-relative notation, e.g. ``models/players/\_humanoid/\_humanoid.gla''.
 Some models, like weapons, have no skeleton - leave this empty in that case.
 
 
 \subsection{GLA Export}
 
 Located in ``File/Export/JA Ghoul 2 skeleton/animation (.gla)'' or invoked by entering ``Export JA Ghoul 2 
 Skeleton \& Animation (.gla)'' in the Search Window opened by default with Space.
 
 \subsubsection*{Settings}
 
 \paragraph*{Base Path}
 See GLM Import Base Path setting.
 
 \paragraph*{gla name}
 The filename of the gla you're exporting. The exporter will probably be able to guess this based on the
 Base Path so you can leave it empty.
 
 \paragraph*{gla reference}
 If you're not creating a completely new skeleton, but merely new animations for an existing one, you have
 to make sure the indices of the bones match. Well, you don't have to make sure yourself, the exporter can
 do that, but you'll have to tell it which skeleton to match (matching done by bone name), and that's what
 this setting is for.
 
 
 \subsection{animation.cfg Export}
 
 Located in ``File/Export/JA Ghoul 2 animation markers (.cfg)'' or invoked by entering ``Export JA Ghoul 2 
 Animation metadata (.cfg)'' in the Search Window opened by default with Space.
 
 GLA files only contain tons of frames. In order for the game to know which ones belong to which animation
 a separate file is used -- the animation.cfg. To automatically create that, add a Time Marker in the Timeline
 (default shortcut: M) and rename it to the name of the animation starting there (default: CTRL+M), then
 use this operator.
 
 \subsubsection*{Settings}
 
 \paragraph*{Offset}
 If you plan to merge your animation with an existing one, the animation starts will be off by the number of
 frames of the file you've merged yours into. That's what this offset is for: It will be added to the frame
 numbers in the exported file.
 
 
\end{document}
